\documentclass[a4paper,11pt]{article}

\usepackage{amsmath}
\usepackage{amssymb}
\usepackage{geometry}
\geometry{margin=25mm}

\title{User's Guide for Stage 2 Disk Evolution Code}
\author{Atsuo Okazaki}
\date{}

\begin{document}
\maketitle

\section{Overview}

This document describes the structure, numerical scheme, and physical
implementation of the Stage~2 version of the one--dimensional viscous
disk evolution code.
Stage~2 corresponds to the first fully time--dependent implementation
of the disk energy equation coupled with irradiation heating including
a finite time delay.

The code is designed for robustness, extensibility, and physical
transparency, and is intended for studies of accretion and decretion
disks in high--energy binary systems.

\section{General Structure of the Code}

The time integration is organized around three nested layers:

\begin{enumerate}
\item \textbf{Output--level time stepping} with a fixed interval
      $\Delta t_{\rm out}$.
\item \textbf{Adaptive substepping} to reach the next output time.
\item \textbf{Local TRY/COMMIT evolution} for each substep.
\end{enumerate}

The main program advances the system by repeatedly calling
\texttt{evolve\_try\_to\_target}, which ensures that the system evolves
from time $t_n$ to $t_n + \Delta t_{\rm out}$ using as many internal
substeps as required for stability.

\section{Main Modules}

\subsection{mod\_global}

This module stores global grid definitions and state arrays.

Key quantities include:
\begin{itemize}
\item Radial grid: \texttt{r(i)}, \texttt{nr}
\item Surface density: \texttt{sigma\_cur(i)}
\item Viscosity: \texttt{nu\_cur(i)}
\item Thermal quantities:
  \texttt{Tmid\_cur}, \texttt{H\_cur}, \texttt{rho\_cur},
  \texttt{kappa\_cur}, \texttt{tau\_cur}
\item Heating and cooling rates:
  \texttt{Qvis\_cur}, \texttt{Qrad\_cur}, \texttt{Qirr\_cur}
\end{itemize}

The suffix \texttt{\_cur} indicates the current committed state inside
a substep.

\subsection{evolve\_substep\_mod}

This module controls the evolution over a single adaptive substep.

The central routine is:
\begin{verbatim}
substep_try_and_commit()
\end{verbatim}

It performs:
\begin{enumerate}
\item Construction of the irradiation profile at the substep start.
\item A TRY evolution using \texttt{evolve\_physics\_one\_substep\_try}.
\item Acceptance or rejection of the step.
\item Commitment of the result to the \texttt{\_cur} arrays.
\end{enumerate}

No retry loop exists inside this routine; retries are handled at a
higher level.

\subsection{evolve\_try\_mod}

This module contains the physics solver for a single substep.

The main routine is:
\begin{verbatim}
evolve_physics_one_substep_try()
\end{verbatim}

It advances:
\begin{itemize}
\item Surface density $\Sigma$
\item Temperature $T$
\item Viscosity $\nu$
\item Disk structure and heating/cooling terms
\end{itemize}

for a given substep size $\Delta t$.

\subsection{evolve\_to\_target (or evolve\_try\_to\_target)}

This routine advances the system from $t_n$ to
$t_n + \Delta t_{\rm out}$ using adaptive substeps.

Algorithm:
\begin{enumerate}
\item Initialize \texttt{t\_local = t\_nd}.
\item While remaining time $>0$:
  \begin{itemize}
  \item Attempt a substep of size $\Delta t_{\rm try}$.
  \item On failure, reduce $\Delta t$.
  \item On success, commit and advance \texttt{t\_local}.
  \end{itemize}
\item Ensure that output times are uniformly spaced.
\end{enumerate}

\section{Governing Equations}

\subsection{Surface Density Evolution}

The disk surface density evolves according to the standard viscous
diffusion equation:
\begin{equation}
\frac{\partial \Sigma}{\partial t}
=
\frac{3}{r} \frac{\partial}{\partial r}
\left[
r^{1/2} \frac{\partial}{\partial r}
\left(
\nu \Sigma r^{1/2}
\right)
\right]
+ S(r,t),
\end{equation}
where $S(r,t)$ is an external source or sink term.

Time integration uses the $\theta$--method:
\begin{equation}
\Sigma^{n+1}
=
\Sigma^{n}
+ \Delta t \left[
(1-\theta) F(\Sigma^n) + \theta F(\Sigma^{n+1})
\right].
\end{equation}

\subsection{Energy Equation}

At each radius, the midplane temperature evolves according to:
\begin{equation}
\Sigma c_V \frac{dT}{dt}
=
Q_{\rm vis} + Q_{\rm irr} - Q_{\rm rad}.
\end{equation}

An implicit Euler step is used:
\begin{equation}
\frac{T^{n+1} - T^n}{\Delta t}
=
\frac{Q_{\rm vis}(T^{n+1}) + Q_{\rm irr} - Q_{\rm rad}(T^{n+1})}
{\Sigma c_V}.
\end{equation}

This nonlinear equation is solved locally using a damped Newton method
with bracketing safeguards.

\section{Convergence and Acceptance Criteria}

Each substep is accepted if:
\begin{itemize}
\item No numerical failure occurs (NaN, divergence).
\item $\Sigma > 0$, $T > 0$, $\nu > 0$ in active cells.
\end{itemize}

The Newton iteration itself enforces local convergence of the energy
equation, so no global thermal balance test is required at this stage.

\section{Irradiation Model}

\subsection{Irradiation Heating Rate}

The irradiation heating rate is computed using Eq.~(16) of
Lee, Okazaki, and Hayasaki (2024):
\begin{equation}
Q_{\rm irr}(\xi)
=
\frac{A_1 L_1}{2\pi r_{\rm in}^2}
\frac{1}{\xi}
\left[
(1+Q_{12}) \frac{dY}{d\xi}
-
\frac{\beta_1 + Q_{12}\beta_2}{\xi^2}
\left(
\frac{Y}{\xi} - \frac{1}{2}\frac{dY}{d\xi}
\right)
\right],
\end{equation}
where $\xi = r/r_{\rm in}$ and $Y = H/r$.

At Stage~2:
\begin{itemize}
\item Shadowing is disabled.
\item $Q_{\rm irr}(r)$ is computed once per substep using the
      geometry at the substep start.
\end{itemize}

\subsection{Time Delay (Scheme C)}

Irradiation is powered by accretion luminosity with a finite delay.

Procedure:
\begin{enumerate}
\item Measure instantaneous accretion rate $\dot M_{\rm in}(t)$ at the
      inner boundary.
\item Store $\dot M_{\rm in}$ in a physical--time history buffer.
\item Define a delay time:
\begin{equation}
t_{\rm delay}
=
f_{\rm delay} \frac{r_{\rm in}^2}{\nu(r_{\rm in})}.
\end{equation}
\item Compute delayed accretion rate
$\dot M_{\rm in}(t - t_{\rm delay})$ by linear interpolation.
\item Set irradiation luminosity:
\begin{equation}
L_{\rm irr}
=
\eta_{\rm acc} \dot M_{\rm in}(t - t_{\rm delay}) c^2,
\end{equation}
optionally capped at the Eddington luminosity.
\end{enumerate}

This scheme ensures causality and avoids instantaneous feedback.

\section{Scope and Limitations of Stage 2}

Stage~2 includes:
\begin{itemize}
\item Fully time--dependent energy equation.
\item Adaptive substepping with fixed output times.
\item Irradiation with time delay.
\end{itemize}

Not included at this stage:
\begin{itemize}
\item Shadowing effects.
\item Implicit irradiation--thermal coupling.
\item Stability branch tracking.
\item Checkpointing of irradiation history buffers.
\end{itemize}

These are intended for later development stages.

\section{Summary}

The Stage~2 code provides a robust and physically consistent framework
for studying irradiated viscous disks with delayed feedback.
It forms a solid foundation for future extensions including shadowing,
branch stability analysis, and global energy diagnostics.

\end{document}
