% IRRADIATION_TREATMENT.tex
% Irradiation treatment for CircumBinary Disks (CBD) and Be disks
% Optically thick and optically thin regions
% ASCII LaTeX only

\documentclass[11pt,a4paper]{article}
\usepackage[utf8]{inputenc}
\usepackage{amsmath,amssymb}
\usepackage{cite}
\usepackage{hyperref}
\usepackage[margin=2.5cm]{geometry}

\title{Irradiation Treatment for CircumBinary and Be Disks}
\author{}
\date{}

\begin{document}
\maketitle

\section{Overview}

This document summarizes the irradiation treatment in the 1D viscous disk code for CircumBinary Disks (CBD) and Be (classical) decretion disks. The thermal balance is governed by
\begin{equation}
  Q_{\mathrm{vis}} + Q_{\mathrm{irr}} = Q_{\mathrm{rad}},
\end{equation}
where $Q_{\mathrm{vis}}$ is viscous heating, $Q_{\mathrm{irr}}$ is irradiation heating (per unit area, from both disk faces), and $Q_{\mathrm{rad}}$ is radiative cooling. The formulae for $Q_{\mathrm{irr}}$ and $Q_{\mathrm{rad}}$ differ between optically thick and optically thin regions.

\section{Irradiation Heating $Q_{\mathrm{irr}}$}

\subsection{Base Formulation: LOH24 Eq.\,(16)}

The code uses the orbit-averaged irradiation flux from Lee, Okazaki \& Hayasaki (2024; LOH24) for a circumbinary disk around a point-like central source. In dimensionless form ($\xi = r/r_{\mathrm{in}}$),
\begin{equation}
  Q_{\mathrm{irr}} = \frac{A_1 L_1}{2\pi r_{\mathrm{in}}^2}\,
  \frac{1}{\xi}\,
  \left[ (1+Q_{12})\,\frac{dY}{d\xi}
       - \frac{\beta_1 + Q_{12}\beta_2}{\xi^2}\,
         \left( \frac{Y}{\xi} - \frac{1}{2}\frac{dY}{d\xi} \right) \right],
  \label{eq:Qirr_LOH24}
\end{equation}
where $Y = H/r$ (aspect ratio), $dY/d\xi$ is the flaring gradient, $A_1 = 1 - \mathrm{albedo}$, $L_1$ is the effective luminosity, and $\beta_1$, $\beta_2$, $Q_{12}$ depend on the binary mass ratio $q$ and gap parameter $C_{\mathrm{gap}}$ (LOH24).

\subsection{CBD Case}

For CircumBinary Disks:
\begin{itemize}
  \item $L_1 = L_{\mathrm{acc}}/(1+q)$ (or similar), with $L_{\mathrm{acc}} = \eta_{\mathrm{acc}} \dot{M} c^2$.
  \item Optional irradiation delay: $L_{\mathrm{irr}}(t) = L_{\mathrm{acc}}(t - \tau)$ where $\tau$ is either explicit (fixed) or viscous ($\tau \propto r_{\mathrm{in}}^2/\nu_{\mathrm{in}}$).
  \item Point-source assumption in Eq.\,(\ref{eq:Qirr_LOH24}); the central binary is treated as a single luminosity source.
\end{itemize}

\subsection{Be Disk Case}

For Be (classical) decretion disks:
\begin{itemize}
  \item $L_1 = L_{\star}$ (stellar luminosity); no accretion-powered irradiation delay.
  \item Finite stellar size is important: the star subtends a significant solid angle at the inner disk.
\end{itemize}

\subsection{Finite-Size Irradiation Source (Chiang \& Goldreich)}

When the central source has finite radius $R_{\star}$, the grazing angle is augmented. Following Chiang \& Goldreich (1997) and LOH24:
\begin{equation}
  \alpha_{\mathrm{total}} = \alpha_{\mathrm{star}} + \alpha_{\mathrm{flare}},
  \label{eq:alpha_total}
\end{equation}
where
\begin{align}
  \alpha_{\mathrm{star}} &= 0.4\,\frac{R_{\star}}{r},
  \label{eq:alpha_star} \\
  \alpha_{\mathrm{flare}} &= \xi\,\frac{dY}{d\xi} = r\,\frac{d(H/r)}{dr}.
  \label{eq:alpha_flare}
\end{align}
The irradiation is scaled:
\begin{itemize}
  \item If $\alpha_{\mathrm{flare}} \ge \alpha_{\mathrm{star}}$:
    $Q_{\mathrm{irr}} \gets Q_{\mathrm{irr}}^{\mathrm{LOH24}} \times
    (\alpha_{\mathrm{star}} + \alpha_{\mathrm{flare}})/\alpha_{\mathrm{flare}}$.
  \item If $\alpha_{\mathrm{flare}} < \alpha_{\mathrm{star}}$:
    $Q_{\mathrm{irr}} = (\alpha_{\mathrm{star}} + \alpha_{\mathrm{flare}})/\alpha_{\mathrm{star}}
    \times Q_{\mathrm{irr}}^{\mathrm{flat,star}}$, with
    $Q_{\mathrm{irr}}^{\mathrm{flat,star}} \propto (R_{\star}/r_{\mathrm{in}})/\xi^3$.
\end{itemize}

\section{Radiative Cooling $Q_{\mathrm{rad}}$}

\subsection{Optically Thick Region ($\tau_R \gtrsim 1$)}

In the diffusion limit (e.g., Kato, Fukue \& Mineshige 2008, Eq.\,3.38):
\begin{equation}
  Q_{\mathrm{rad}}^{\mathrm{thick}} = \frac{64\,\sigma_{\mathrm{SB}} T^4}{3\,\kappa_R\,\Sigma},
  \label{eq:Qrad_thick}
\end{equation}
where $\kappa_R$ is the Rosseland mean opacity, $\Sigma$ is the surface density, and $\tau_R = \kappa_R\Sigma/2$ (one-side optical depth). The Rosseland mean is appropriate for diffusion through an optically thick medium.

\subsection{Optically Thin Region ($\tau_R \ll 1$)}

In the optically thin limit, cooling is dominated by emission. D'Alessio et al.\ (1998) show that the cooling rate scales with the Planck mean opacity $\kappa_P$:
\begin{equation}
  \Lambda_d = 4\pi\kappa_P\rho\left( \frac{\sigma_{\mathrm{SB}}T^4}{\pi} - J_d \right),
  \label{eq:Lambda_DAlessio}
\end{equation}
which in the optically thin limit gives $Q_{\mathrm{rad}} \propto \kappa_P \Sigma \sigma_{\mathrm{SB}} T^4$. The implemented form (per unit area, both faces) is
\begin{equation}
  Q_{\mathrm{rad}}^{\mathrm{thin}} = 2\,\kappa_P\,\Sigma\,\sigma_{\mathrm{SB}}\,T^4.
  \label{eq:Qrad_thin}
\end{equation}
Planck mean opacity is used because emission weights the spectrum by the Planck function, unlike the Rosseland mean used for radiative diffusion.

\subsection{Bridging Between Thick and Thin}

A smooth transition is used:
\begin{equation}
  Q_{\mathrm{rad}} = \frac{1}{\displaystyle \frac{1-w}{Q_{\mathrm{rad}}^{\mathrm{thin}}}
    + \frac{w}{Q_{\mathrm{rad}}^{\mathrm{thick}}}},\quad
  w = \frac{\tau_R}{1 + \tau_R}.
  \label{eq:Qrad_bridge}
\end{equation}

\section{Opacity Tables}

\subsection{Rosseland Mean $\kappa_R$}

OPAL, Ferguson, AESOPUS, Semenov et al.\ (2003) tables are used, depending on temperature and density.

\subsection{Planck Mean $\kappa_P$: Calculation Method}

The Planck mean opacity $\kappa_P$ is required for optically thin radiative cooling (Eq.\,\ref{eq:Qrad_thin}). The calculation switches at a critical temperature:

\paragraph{Temperature $T \le 10^4\,\mathrm{K}$:} Semenov et al.\ (2003) Planck mean table.
\begin{itemize}
  \item File: \texttt{data/SemenovPlanckTable.data}
  \item Grid: $\log_{10}(T/\mathrm{K})$ from 1.0 to 4.0 (i.e., $10\,\mathrm{K}$ to $10^4\,\mathrm{K}$), $\log_{10}(\rho/\mathrm{g}\,\mathrm{cm}^{-3})$ from $-17$ to $-7$
  \item Values stored as $\log_{10}(\kappa_P/\mathrm{cm}^2\,\mathrm{g}^{-1})$
  \item 2D spline (or polynomial) interpolation in $(\log T, \log\rho)$
\end{itemize}
The Semenov Planck table covers dust + gas opacity and is appropriate for protoplanetary and cool disk conditions.

\paragraph{Temperature $T > 10^4\,\mathrm{K}$:} Analytic formula (Kramers free-free + electron scattering).
Above $10^4\,\mathrm{K}$, dust is destroyed and the opacity is dominated by free-free absorption and electron scattering. The Planck mean is computed as
\begin{equation}
  \kappa_P = \kappa_{\mathrm{ff}} + \kappa_{\mathrm{es}},
  \label{eq:kappaP_split}
\end{equation}
where
\begin{align}
  \kappa_{\mathrm{ff}} &= 3.68\times 10^{22}\,g_{\mathrm{ff}}\,(1-Z)(1+X)\,\rho\,T^{-7/2},
  \label{eq:kappa_ff} \\
  \kappa_{\mathrm{es}} &= 0.2\,(1+X).
  \label{eq:kappa_es}
\end{align}
Implemented parameters: $g_{\mathrm{ff}} = 1$, $X = 0.7$, $Z = 0.02$ (cosmic abundance); $\kappa_{\mathrm{es}} \approx 0.34\,\mathrm{cm}^2\,\mathrm{g}^{-1}$. The free-free opacity follows the Kramers approximation; units are CGS ($\rho$ in $\mathrm{g}\,\mathrm{cm}^{-3}$, $T$ in K, $\kappa_P$ in $\mathrm{cm}^2\,\mathrm{g}^{-1}$).

\paragraph{Summary:}
\begin{center}
\begin{tabular}{ll}
  \hline
  $T \le 10^4\,\mathrm{K}$ & Semenov et al.\ (2003) table interpolation \\
  $T > 10^4\,\mathrm{K}$   & $\kappa_P = \kappa_{\mathrm{ff}} + \kappa_{\mathrm{es}}$ (Eqs.\,\ref{eq:kappa_ff}--\ref{eq:kappa_es}) \\
  \hline
\end{tabular}
\end{center}
This choice is motivated by the fact that published Planck mean tables (e.g., Semenov) typically extend only to $T \sim 10^4\,\mathrm{K}$, while Be disks and inner CBD regions can reach slightly higher temperatures; the analytic extension ensures physically reasonable cooling in those regimes.

\section{Optically Thin Irradiation (Chiang \& Goldreich Eq.\,12)}

Chiang \& Goldreich (1997) derive three regimes for the incident stellar flux in optically thin surface layers (Eq.\,12a--c):
\begin{itemize}
  \item Regime (a): optically thin to stellar radiation; flux penetrates and heats the whole column.
  \item Regime (b): intermediate; partial absorption.
  \item Regime (c): optically thick to stellar radiation; absorption at surface (standard treatment).
\end{itemize}
The current code applies the same $Q_{\mathrm{irr}}$ profile (LOH24 + finite-star correction) to both optically thick and thin cells. This corresponds to a single effective regime (c-like for thick, and an approximation for thin). For outer disk regions where $\tau_R < 1$ but stellar irradiation is still absorbed, the adopted $Q_{\mathrm{irr}}$ remains a reasonable approximation; true limb-to-limb optically thin irradiation would require a regime-dependent treatment per Chiang \& Goldreich Eq.\,12a--c.

\section{Summary by Case}

\begin{table}[h]
\centering
\begin{tabular}{l|l|l}
  \hline
  & CBD & Be Disk \\
  \hline
  $L_{\mathrm{irr}}$ & $L_{\mathrm{acc}}$ (optionally delayed) & $L_{\star}$ \\
  Finite star & Optional ($\alpha_{\mathrm{star}}$) & Recommended ($\alpha_{\mathrm{star}}$) \\
  $Q_{\mathrm{irr}}$ & LOH24 Eq.\,16 + Eq.\,(\ref{eq:alpha_total}) & Same \\
  Optically thick $Q_{\mathrm{rad}}$ & Eq.\,(\ref{eq:Qrad_thick}), $\kappa_R$ & Same \\
  Optically thin $Q_{\mathrm{rad}}$ & Eq.\,(\ref{eq:Qrad_thin}), $\kappa_P$ & Same \\
  \hline
\end{tabular}
\caption{Irradiation and cooling treatments by disk type.}
\end{table}

\section{Implementation Guide}

This section provides code file paths and input parameters for readers who wish to locate or modify the implementation.

\subsection{Code Files}

\begin{center}
\begin{tabular}{ll}
  \hline
  File & Role \\
  \hline
  \texttt{irradiation\_mod.f90} & $Q_{\mathrm{irr}}$ (LOH24 Eq.\,16), finite-star correction, \\
  & $L_{\mathrm{irr}}$ from accretion or $L_{\star}$, irradiation delay buffer \\
  \texttt{disk\_thermal\_mod.f90} & Thermal balance: \texttt{heating\_cooling\_cell}, \\
  & $Q_{\mathrm{vis}}$, $Q_{\mathrm{rad}}$ (thick/thin bridging), $\kappa_R$, $\kappa_P$ usage \\
  \texttt{opacity\_table\_mod.f90} & \texttt{get\_opacity\_Planck\_rhoT}, Semenov Planck table, \\
  & Kramers+$\kappa_{\mathrm{es}}$ for $T > 10^4\,\mathrm{K}$ \\
  \texttt{disk\_energy\_mod.f90} & Thermal solver, iteration over $T_{\mathrm{mid}}$ and $Q_{\mathrm{irr}}$ \\
  \texttt{output\_mod.f90} & Disk structure output, $Q_{\mathrm{rad}}$ for diagnostics \\
  \texttt{mod\_global.f90} & Global flags: \texttt{use\_irradiation}, \texttt{use\_be\_decretion}, etc. \\
  \texttt{setup.f90} & Namelist input (\texttt{/disk\_mode/}) \\
  \hline
\end{tabular}
\end{center}

\subsection{Key Input Parameters}

Namelist \texttt{/disk\_mode/} (in \texttt{ad1d.in} or equivalent):
\begin{center}
\begin{tabular}{llp{7cm}}
  \hline
  Parameter & Default & Description \\
  \hline
  \texttt{use\_irradiation} & \texttt{.false.} & Enable irradiation heating \\
  \texttt{use\_be\_decretion} & \texttt{.false.} & Be disk mode; $L_{\mathrm{irr}} = L_{\star}$, no delay \\
  \texttt{use\_irradiation\_delay} & \texttt{.false.} & Delay $L_{\mathrm{irr}}$ by $\tau$ (CBD accretion) \\
  \texttt{use\_finite\_irradiation\_source} & \texttt{.false.} & Add $\alpha_{\mathrm{star}}$ (Eq.\,\ref{eq:alpha_star}) \\
  \texttt{tau\_irr\_lag\_mode} & \texttt{'explicit'} & \texttt{'explicit'}: fixed $\tau$; \texttt{'viscous'}: $\tau \propto r_{\mathrm{in}}^2/\nu_{\mathrm{in}}$ \\
  \texttt{tau\_irr\_lag\_nd} & 0 & \texttt{explicit}: delay in units of $t_0$; \texttt{viscous}: prefactor \\
  \hline
\end{tabular}
\end{center}

Related parameters (in \texttt{/star\_params/}, \texttt{/scale\_params/}, or elsewhere): \texttt{R\_star}, \texttt{L\_star}, \texttt{q} (mass ratio), \texttt{f\_edd\_cap} (cap $L_{\mathrm{acc}}/L_{\mathrm{Edd}}$). Albedo is hardcoded in \texttt{irradiation\_mod.f90} (\texttt{albedo = 0.9}).

\section{References}

\begin{enumerate}
  \item Chiang, E.\,I., \& Goldreich, P.\,1997, ApJ, 490, 368\\
  ``Spectral Energy Distributions of T Tauri Stars with Passive Circumstellar Disks''
  \item D'Alessio, P., Canto, J., Calvet, N., \& Lizano, S.\,1998, ApJ, 500, 411\\
  ``Accretion Disks around Young Objects. I.''
  \item Kato, S., Fukue, J., \& Mineshige, S.\,2008, {\it Black-Hole Accretion Disks}\\
  Kyoto University Press
  \item Lee, Y., Okazaki, A.\,T., \& Hayasaki, K.\,2024, ApJ, 975, 65 (LOH24)\\
  ``Circumbinary disk spectra irradiated by two central accretion disks in a binary black hole system''
  \item Semenov, D., Henning, T., Helling, C., Ilgner, M., \& Sedlmayr, E.\,2003, A\&A, 410, 611\\
  ``Rosseland and Planck mean opacities for protoplanetary discs''
\end{enumerate}

\end{document}
